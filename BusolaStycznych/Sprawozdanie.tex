\documentclass[a4paper,11pt]{article} 
\usepackage{fancyhdr}
\usepackage{polski}
\usepackage{graphicx}
\usepackage[left=2.5cm,top=2.5cm,right=2.5cm,bottom=4.5cm]{geometry}
\usepackage[T1]{fontenc}
\usepackage{amsfonts}
\usepackage[utf8]{inputenc}
\usepackage{enumerate}
\usepackage{indentfirst}
\usepackage{lipsum}
\usepackage{multirow}
\usepackage{multicol}
\usepackage[export]{adjustbox}
\usepackage[font=small,labelfont=bf]{caption} 
\usepackage{booktabs}
\usepackage{amsmath}
\usepackage{float}

\input{variables.txt}

\pagestyle{fancy}
\renewcommand{\headrulewidth}{1pt} 
\fancyhead[R]{\includegraphics[height=80pt]{logo.png}
}
\lhead{Akademia Górniczo-Hutnicza \\ im. Stanisława Staszica w Krakowie\\  Wydział \Wydzial}
\setlength{\headheight}{80pt}

\cfoot{}
\rfoot{\thepage} 


\begin{document}
%\maketitle

%Tabelke umiescic w osobnym pliku
\input{table.txt}
\vspace{1cm}
\begin{center}
	\Huge{\textbf{\Temat}} \\
	\vspace{0.5cm}
	\Large{\textbf{Ćwiczenie nr \NrCwiczenia}} \\
	\vspace{0.5cm}
	\large{\PierwszyAutor} \\
	\large{\DrugiAutor} \\
	%\large{\TrzeciAutor}\\
	\vspace{1cm}
\end{center}

\tableofcontents

\newpage 
\section{Wstęp}

\input{1wstep.txt}

\section{Układ pomiarowy}

\input{2UkladPomiarowy.txt}

%\section{Przebieg doświadczenia}

%\input{3Przebieg.txt}

\section{Wyniki pomiarów}

\input{4Wyniki.txt}

\input{5OpracowanieWynikowPomiaru.txt}

\section{Wnioski}

\input{6Wnioski.txt}

\end{document}