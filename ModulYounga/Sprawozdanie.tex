\documentclass[a4paper,11pt]{article} 
\usepackage{fancyhdr}
\usepackage{polski}
\usepackage{graphicx}
\usepackage[left=2.5cm,top=2.5cm,right=2.5cm,bottom=4.5cm]{geometry}
\usepackage[T1]{fontenc}
\usepackage{amsfonts}
\usepackage[utf8]{inputenc}
\usepackage{enumerate}
\usepackage{indentfirst}
\usepackage{lipsum}
\usepackage[export]{adjustbox}
\usepackage[font=small,labelfont=bf]{caption} 
\usepackage{booktabs}

\pagestyle{fancy}
\renewcommand{\headrulewidth}{1pt} 
\fancyhead[R]{\includegraphics[height=80pt]{log.png}
}
\lhead{Akademia Górniczo-Hutnicza \\ im. Stanisława Staszica w Krakowie\\  Wydział Informatyki }
\setlength{\headheight}{80pt}

\cfoot{}
\rfoot{\thepage} 


%%%%%%%%%%%%%%%%%%%%%%%%%DADOS%%%%%%%%%%%%%%%%%%%%%%%%%%%%%
\begin{document}
%\maketitle

%Tabelke umiescic w osobnym pliku
\input{table.txt}

\section{Cel ćwiczenia}

Celem ćwiczenia było

\section{Potrzebne przyrządy}


\section{Wykonywanie ćwiczenia}

\begin{table}[h]
\centering
\begin{tabular}{|l|l|l|}
\hline
Numer pomiaru & Czas 10 okresów {[}s{]} & Okres T {[}s{]} \\ \hline
1.            & 8,62                    & 0,862                 \\ \hline
2.            & 8,59                    & 0,859                 \\ \hline
3.            & 9,01                    & 0,901                 \\ \hline
4.            & 8,44                    & 0,844                 \\ \hline
5.            & 8,53                    & 0,853                 \\ \hline
6.            & 8,50                    & 0,850                 \\ \hline
7.            & 8,76                    & 0,876                 \\ \hline
8.            & 8,51                    & 0,851                 \\ \hline
9.            & 8,57                    & 0,857                 \\ \hline
10.           & 8,71                    & 0,871                 \\ \hline
\end{tabular}
\end{table}


\end{document}